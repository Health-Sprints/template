\documentclass{article}

\begin{document}

\title{7 Vorgehen bei nachgewiesener Infektion}

\maketitle


Ist ein Fall unter SuS nachgewiesen worden, so ist die gesamte Klasse/Kurs/Lernverband - also alle Personen-(gruppen), zu denen eine relevante Exposition (> 30 Minuten, in einem nicht ausreichend belüfteten Raum, siehe (21) (22) (23) (24)) bestand, als Kontaktpersonen der Kategorie 1 (KP1)\footnote{https://www.rki.de/DE/Content/InfAZ/N/Neuartiges\_Coronavirus/Kontaktperson/Management.html} zu betrachten und entsprechend zu verfahren, d.h. sofortige Quarantäne, bzw. Isolierung bei bestehender Symptomatik. 


Bei nachgewiesenen Infektionen des Lehr- und Betreuungspersonals gelten analog alle Personengruppen (Klassen, Kurse) mit relevanter Exposition als KP1. 


Die Durchführung einer Reihenuntersuchung (Screening) bei hohem Infektionsgeschehen und/oder die Schließung der gesamten Einrichtung bei Ausbrüchen liegen im Ermessen der lokalen Behörden.

\end{document}
