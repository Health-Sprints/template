\documentclass{article}

\begin{document}

\title{copy of 6 Testkriterien }

\maketitle


Die Kriterien für die Testindikation können in drei Kategorien unterschieden werden, 1. Vulnerabilität der betroffenen Person oder deren Kontaktpersonen; 2. die klinische Symptomatik; 3. die Expositionswahrscheinlichkeit einmal individuell und dann grundsätzlich basierend auf der Häufigkeit von COVID-19 Fällen in der Region 

\begin{enumerate}
\item Vulnerabilität der betroffenen Person: 

\begin{itemize}
\item  erhöhtes Risiko für einen schweren Verlauf 


\item  enger Kontakt zu Personen, die einer Risikogruppe angehören (bspw. Familie, Haushalt, Schule) 


\end{itemize}

\item klinische Symptomatik: 

\begin{itemize}
\item ARE: klinische Symptome wie Husten, Fieber, Schnupfen, mit oder ohne Fieber (> 38°C) (20) 


\item Störung des Geruchs- und/oder Geschmackssinns (Hypo- oder Anosmie bzw. Hypo- oder Ageusie) 


\item Speziell bei Kindern: Gastrointestinale Symptome (Durchfall, Erbrechen), Myalgie (20)


\end{itemize}

\item Expositionswahrscheinlichkeit:

\begin{itemize}
\item Kontakt zu Personen mit nachgewiesener SARS-CoV-2 Infektion


\item Kontakt im Haushalt oder zu einem Cluster von Personen mit ARE ungeklärter Ursache 


\item Link zu einem bekannten Ausbruchsgeschehen 


\item Rückkehr aus einem Risikogebiet oder Gebiet mit hoher lokaler Inzidenz 


\item weiterhin enger Kontakt zu vielen Menschen (bspw. Lehr- und Betreuungspersonal)


\end{itemize}

\end{enumerate}

\subsection{6.1 Anwendung der Testkriterien }\label{H6515269}



Das Erfüllen eines Kriteriums bedeutet nicht, dass zwingend ein Test zu erfolgen hat. Vielmehr sollen die Kriterien helfen, die geeignete Maßnahme (Test, Isolierung oder Quarantäne) anzuwenden. Im folgenden Abschnitt ist dargestellt, welche Kriterien für die Durchführung eines Tests erfüllt sein sollten. 


\subsubsection{Fall-basiertes Testen }\label{H8802528}



Indikationen für eine Testung ergeben sich entweder für symptomatische SuS und Schulpersonal sofern ein hinreichendes klinisches Bild vorliegt und/oder ein epidemiologischer Zusammenhang zu einem Infektionsgeschehen oder einer vulnerablen Gruppe besteht. 


\emph{Ein Test ist durchzuführen, wenn mindestens eines der folgenden Kriterien erfüllt ist:}


Schwere respiratorische Symptome\footnote{Bei beschriebener Symptomatik sollte neben einem SARS-CoV-2 Test ebenfalls ein Influenza Test durchgeführt werden.} (bspw. durch akute Bronchitis, Pneumonie, Atemnot oder Fieber) 

\begin{itemize}
\item Akute Hypo- oder Anosmie bzw. Hypo- oder Ageusie (Störung des Geruchs- bzw. Geschmackssinns) 


\item Anhaltende akute respiratorische Symptome jeder Schwere über einen Zeitraum von > 5 Tagen ohne Verbesserung 


\item Ungeklärte Erkrankungssymptome und Kontakt (KP1) zu einer Person mit bestätigter SARS-CoV-2 Infektion 


\item Akute respiratorische Symptome jeder Schwere, insbesondere bei:

\begin{itemize}
\item Personen, die einer Risikogruppe angehören ODER 


\end{itemize}
\begin{itemize}
\item Erhöhter Expositionswahrscheinlichkeit, bspw. im Rahmen eines bekannten Ausbruchs, einer Veranstaltung außerhalb der Schule mit > 10 Personen in geschlossenen und unzureichend durchlüfteten Räumen und unzureichender Anwendung der AHA+A+LRegeln ODER 


\item Kontakt im Haushalt oder zu einem Cluster von Personen mit ARE ungeklärter Ursache UND eine erhöhte COVID-19 7-Tages-Inzidenz ODER 


\item Schulpersonal mit weiterhin engem Kontakt zu vielen Menschen (SuS) und/oder zu Personen, die einer Risikogruppe angehören (auch außerhalb der Schule)


\end{itemize}

\item Klinische Verschlechterung bei bestehender Symptomatik


\end{itemize}

Zur Erklärung, ob die Kriterien erfüllt sind, die im Zusammenhang mit „Akute respiratorische Symptome jeder Schwere“ abgefragt werden, empfehlen sich standardisierte Fragen, die eine schnelle Beurteilung ermöglichen:

\begin{enumerate}
\item Gehört die Person zu einer Risikogruppe oder hat Kontakt zu Personen, die einer Risikogruppe angehören? 


\item Haben Familienmitglieder regelmäßig Kontakt zu Personen, die einer Risikogruppe angehören, innerhalb oder außerhalb der Familie, z.B. ein Elternteil ist in der Altenpflege tätig. 


\item Gibt es aktuell ungeklärte akute Erkrankung(en) in der Familie? 


\item Besteht individuell ein erhöhtes Infektions- oder Weiterverbreitungsrisiko, z. B. aufgrund einer Teilnahme an einer Großveranstaltung innerhalb der letzten 1-2 Wochen? 


\item Handelt es sich um Lehr- oder Betreuungspersonal oder ist anderweitig von weiterhin vielen relevanten Kontakten auszugehen? 


\end{enumerate}

\subsubsection{Maßnahmen bei Symptomen, auch ohne Vorliegen eines Testergebnisses Da jegliche respiratorische}\label{H1505935}



Symptomatik, auch ein alleiniger Schnupfen, Ausdruck einer SARS-CoV-2- Infektion sein kann, sollten Personen, bei denen die Testkriterien nicht erfüllt sind oder kein Testergebnis vorliegt, sich trotzdem so verhalten, dass Übertragungen verhindert werden, wenn sie eine COVID-19- Erkrankung hätten. Dazu gehört, eine Isolation zu Hause für 5 Tage UND mindestens 48 h Symptomfreiheit vor Beendigung sowie eine Kontaktreduktion. Bei sekundärer klinischer Verschlechterung ist eine sofortige Testung auf SARS-CoV-2 empfohlen.

\end{document}
