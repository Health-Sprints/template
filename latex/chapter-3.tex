\documentclass{article}

\begin{document}

\title{ 3 Infektionsepidemiologische Grundannahmen und Beobachtungen zu Schulen}

\maketitle


Folgende Aspekte sind nach derzeitigem Wissensstand hinsichtlich der Maßnahmenempfehlung in Schulen in Zeiten der COVID-19 Pandemie von Bedeutung (zu epidemiologischen Daten s. Lageberichte des RKI (1), zu Studienübersichten s. auch 2. Quartalsbericht der Corona-Kitastudie\footnote{https://www.rki.de/DE/Content/InfAZ/N/Neuartiges\_Coronavirus/Projekte\_RKI/KiTa-Studie-Berichte/KiTaStudie\_QuartalIV\_2020.pdf} sowie der Technical Report des ECDC (2)):

\begin{itemize}
\item Schülerinnen und Schüler (SuS) sind prinzipiell empfänglich für eine Infektion mit SARS-CoV-2und können andere infizieren (3).


\item Kinder und jüngere Jugendliche sind jedoch seltener betroffen als Erwachsene (3).


\item Mit zunehmendem Alter ähneln Jugendliche hinsichtlich Empfänglichkeit und Infektiosität den Erwachsenen (4) (5) (6) (7) . Epidemiologisch folgen die Infektionen bei Kindern bisher dem Infektionsgeschehen bei Erwachsenen (8).


\item Kinder und Jugendliche zeigen häufig keine oder nur eine milde Symptomatik (9) (10) .


\item  Im Erkrankungsfall erkranken Kinder und Jugendliche in aller Regel leicht. Dies trifft nach Einschätzung pädiatrischer Fachgesellschaften (11) auch bei Vorliegen von aus dem Erwachsenenalter bekannten Risikofaktoren/chronischen Erkrankungen zu, sofern diese gut kompensiert bzw. behandelt sind. Das individuelle Risiko bei Vorliegen von Vorerkrankungen unterliegt einer ärztlichen Einzelfallbeurteilung, unter Berücksichtigung der Empfehlungen und Stellungnahmen der pädiatrischen Fachgesellschaften für das jeweilige Krankheitsbild.


\item Schwere Verläufe sind im Kindes- und Jugendalter selten (deutlich seltener als bei Erwachsenen), ebenso wie Todesfälle (1; 9) (10). Allerdings werden auch für das Kindesalter länger anhaltende Krankheitssymptome beschrieben und der Anteil der Spätfolgen ist bisher noch nicht bekannt (12).


\item Die anerkannten Infektionsschutzmaßnahmen sind auch im Kindes- und Jugendalter wirksam und ein wichtiger Baustein bei der Bewältigung der Pandemie (siehe Leitlinie: Maßnahmen zur Prävention und Kontrolle der SARS-CoV-2-Übertragung in Schulen\protect\footnotemark{} ).


\item Ausbrüche in Schulen kommen auch in Bildungseinrichtungen vor (13). Sie können bislang gut kontrolliert werden. Größere Ausbruchsgeschehen sind die Ausnahme. Oftmals erfolgt der Eintrag in Schulen über Erwachsene.


\item Das Ausmaß einer Übertragung innerhalb der Schulen und von den Schulen in die Familien/Haushalte ist Gegenstand der Forschung.


\item Aufgrund des zunehmenden Anteils von besorgniserregenden Varianten von SARS-CoV-2 auch in Deutschland, die mit einem erhöhten Ansteckungspotenzial einhergehen, ist auch mit einem erhöhten Übertragungswahrscheinlichkeit in Schulen zu rechnen (14) (15) (16) (17).


\item Aufgrund des zunehmenden Anteils von besorgniserregenden Varianten von SARS-CoV-2 auch in Deutschland, die mit einem erhöhten Ansteckungspotenzial einhergehen, ist auch mit einem erhöhten Übertragungswahrscheinlichkeit in Schulen zu rechnen (14) (15) (16) (17). („Fernfeld“). Das Risiko einer Übertragung über das Fernfeld erhöht sich bei besonders starker Partikelemission (Singen oder Schreien), besonders langem Aufenthalt der infektiösen Person(en) in einem gegebenen Raum und unzureichender Lüftung/Frischluftzufuhr.


\end{itemize}\addtocounter{footnote}{-1}\stepcounter{footnote}
\footnotetext{Maßnahmen zur Prävention und Kontrolle der SARS-CoV-2-Übertragung in Schulen - Lebende Leitlinie https://www.awmf.org/leitlinien/detail/ll/027-076.html}



\end{document}
