\documentclass{article}

\begin{document}

\title{4 Ziele der Empfehlungen}

\maketitle


Zu den vorrangigen Zielen der Testkriterien im Schulkontext gehören

\begin{enumerate}
\item den Schulbetrieb (Wechsel-/Präsenzunterricht) an die jeweilige Situation in der Gesamtbevölkerung anzupassen und durch angepasste Maßnahmen kontinuierlich und dauerhaft aufrecht zu erhalten, 


\item Ausbrüche an Schulen zu verhindern, früh zu erkennen und effektiv einzudämmen, 


\item Fälle mit erhöhtem Risiko für einen schweren Verlauf rechtzeitig einer Therapie zuzuführen, 


\item Erkrankungsfälle mit Kontakt zu Personen, die einer Risikogruppe angehören (SuS, Familienangehörige) früh zu identifizieren um eine Ansteckung dieser Kontaktpersonen zu verhindern, 


\item Fälle mit verstärkter Exposition gegenüber einer größeren Anzahl weiterer Personen früh zu erkennen und 


\item Verbreitung prospektiv zu verhindern. 


Einschränkung: Es ist nicht das Ziel, alle Fälle unter SuS und Schulpersonal zu identifizieren. Vielmehr bleibt zu beachten, dass ein Teil der Infektionen (unabhängig von den ergriffenen Maßnahmen) weiterhin unerkannt bleibt. Es gilt, dass der Schutz von Personen, die einer Risikogruppe angehören, eine besondere Priorität hat.





\end{enumerate}
\end{document}
