\documentclass{article}

\begin{document}

\title{5 Grundüberlegungen}

\maketitle


Für die Entwicklung der Testkriterien gelten die folgenden Überlegungen: 

\begin{enumerate}
\item Testkapazitäten sollen effizient eingesetzt werden. 


\item Testen dient nicht der Bestätigung aller COVID-19-Fälle an Schulen in Deutschland. 


\item Wird kein Test durchgeführt, impliziert das nicht, dass die Person kein COVID-19 hat und nicht (selbst-)isoliert oder enge Kontaktpersonen quarantänisiert werden sollten. 


\item Das klinische Bild von COVID-19 ist (gerade bei Kindern und Jugendlichen (19)) vielfältig UND kann anhand der klinischen Symptome nicht von anderen ARE unterschieden werden. Es gibt jedoch spezifische Symptome, die einen hohen Vorhersagewert für eine COVID-19-Erkrankung haben, wie die Störung des Geruchs- und Geschmackssinns, welche sich jedoch kaum objektivieren lassen. 


\item Alle Personen mit respiratorischen Symptomen können potenziell an COVID-19 erkrankt sein. Auch andere akute Atemwegsinfektionen können zu Ausbrüchen und damit zur Unterbrechung eines kontinuierlichen Schulbetriebs führen. Falls ein Testergebnis nicht zeitnah vorliegt, sollten Erkrankte daher den empfohlenen Verhaltensregeln folgen (z.B. Selbstisolierung).


\end{enumerate}

\subsection{5.1 Hintergrund }\label{H5129027}



Die höchste Priorität der Testung liegt auf symptomatischen Personen in Abhängigkeit von der lokalen/regionalen Situation, dem individuellen Expositionsrisiko oder der Zugehörigkeit zu einer Person, die einer Risikogruppe angehört bzw. dem engen Kontakt zu solchen Personen in Familie, privatem Umfeld oder durch die berufliche Tätigkeit oder Zugehörigkeit zu einer Gruppe mit Risiko zu häufiger Transmission (wie z.B. Lehrern und Lehrerinnen) . 


Die nachfolgenden Testkriterien sollen die Entscheidungsfindung der Gesundheitsämter und behandelnden Ärztinnen und Ärzte, ob getestet wird oder ob Isolierung der betroffenen Person (einschließlich der Quarantäne von ansteckungsverdächtigen engen Kontaktpersonen) in Abhängigkeit von der aktuellen Situation (z.B. lokales Infektionsgeschehen) und den bereits durchgeführten Präventionsmaßnahmen\footnote{Im Sinne der Empfehlungen im Dokument: Präventionsmaßnahmen in Schulen während der COVID-19 Pandemie} unterstützen. 


Bei jedem positiven Testergebnis ist eine anschließende ärztliche Aufklärung und Betreuung, sowie die Einleitung einer Kontaktpersonennachverfolgung durch das zuständige Gesundheitsamt unablässig. 


Bei (anhaltenden) Symptomen, entsprechender Symptomatik oder ggf. bei negativem Testergebnis auf SARS-CoV-2 ist auch differentialdiagnostisch eine Testung auf andere, der individuellen Symptomatik entsprechenden Erkrankungen empfohlen.

\end{document}
